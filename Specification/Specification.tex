\documentclass{article}
\usepackage{fontspec}
\usepackage[hmargin=1.25in,vmargin=1in]{geometry}
\setmainfont{微软雅黑}
\XeTeXlinebreaklocale "zh"
\XeTeXlinebreakskip = 0pt plus 1pt minus 0.1pt
\begin{document}
\title{程序说明}
\date{}
\maketitle

\section{程序结构}
\ \\
\begin{table}[htbp]
\begin{tabular}{|l|l|}
\hline
\textbf{data\_in}  & 放置输入初始数据文件RHO,U,P,config的文件夹\\
\hline
\textbf{data\_out} & 放置输出计算过程中RHO,U,P,E和拉式坐标X变化结果的文件夹\\
\hline
\textbf{file\_io}  & 数据文件读入与读出的程序\\
\hline
\textbf{finite\_difference\_solver} & 有限格式的算法程序\\
\hline
\textbf{Riemann\_solver} & 精确Riemann解法器\\
\hline
\textbf{LAG\_source.c}   & 主程序\\
\hline
\textbf{make.sh}         & 编译和运行程序的脚本\\
\hline
\end{tabular}
\end{table}

在终端下运行make.sh即可使用gcc编译,使用MATLAB程序画图。

\section{格式}
\ \\

精确Riemann解法器根据[1]所写,$u^{*},p^{*}$为Rienmann问题$*$区域的解。

格式为向前Euler:(GRP/Gdounov\_solver\_source.c)

$m_i(1/\rho_i^{n+1}-1/\rho_i^{n})-\Delta t(u_{i+1/2}^n-u_{i-1/2}^n)=0$,

$m_i(u_i^{n+1}-u_i^{n})+\Delta t(p_{i+1/2}^n-p_{i-1/2}^n)=0$,

$m_i(e_i^{n+1}-e_i^{n})+\Delta t(p_{i+1/2}^nu_{i+1/2}^n-p_{i-1/2}^nu_{i-1/2}^n)=0$。

接触间断的位置计算:

$x_{i+1/2}^{n+1}=x_{i+1/2}^{n}+\Delta tu_{i+1/2}^{n+1/2}$。

节点$j+1/2$时平均数值通量:

$u_{i+1/2}^{n+1/2}=u_{i+1/2}^{*n}+\frac{\Delta t}{2}(\frac{D u}{D t})_{i+1/2}^{n}$,

$p_{i+1/2}^{n+1/2}=p_{i+1/2}^{*n}+\frac{\Delta t}{2}(\frac{D p}{D t})_{i+1/2}^{n}$。

拉式Gdounov格式$(\frac{D u}{D t})_{i+1/2}^{n},(\frac{D p}{D t})_{i+1/2}^{n}$取0。

拉式GRP格式由斜率求接触间断处的物质导数(GRP\_solver\_source.c),由GRP格式算出时间导数(linear\_GRP\_solver\_LAG.c),向前Euler,再通过斜率限制器得到斜率(GRP\_solver\_source.c)。

\section{算例}
\ \\

6.1 sod激波管,

6.2.1 Shock-Contact Interaction。

画的图为转换到欧拉坐标下的RHO,U,P以及拉式坐标的图。


\section*{Refences}
[1] E. F. Toro. A Fast Riemann Solver with Constant Covolume Applied to the
Random Choice Method. Int. J. Numer. Meth. Fluids, 9:1145–1164, 1989.

\end{document}
